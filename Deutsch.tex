\documentclass[a5paper]{book}
\usepackage[utf8]{inputenc}
\usepackage[german]{babel}
\usepackage[a5paper]{geometry}

\title{Mein Tagebuch}
\author{Helene Martha Schaarschmidt}
\date{19. Januar 1959} 

\begin{document}
\maketitle

\newpage
\setcounter{page}{1}
%%% Seite 1 %%%
Ich, Helene Martha Penzel, wurde als die Tochter des Klempners August Penzel am 15. November 1903 in Plauen geboren. 
Zurzeit wohnten meine Eltern in der Friedrichstraße.
Meine ersten Jahre verlebte ich aber dann in der Moritzstraße. 
Dort bekam ich eine kleine Schwester, welche aber wieder gestorben ist. 
Später zogen wir in die König-Georgstraße 36.
Im Jahre 1910, am 31. Oktober, bekam ich einen Bruder. 
Vier Jahre darauf brach der furchtbare Weltkrieg aus.
Am 1. August 1914 war Mobilmachungstag. 
Am 3. August musste mein lieber Vater schon fort.

%%% Seite 2 %%%
Es kamen dann furchtbare Zeiten.
Mein Vater war immer in Russland.
Im Jahre 1918 wurde ich dann konfirmiert. 
Mein Vater konnte nicht heimkommen. 
Ich lernte dann als Verkäuferin bei der Firma S. Winter.
Später war ich bei Fr. Egerland als Sarniererin tätig.
Im Jahre 1920 besuchte ich dann die Tanzstunde bei J. L. Nuss, wo ich schöne Stunden verlebte mit Herrn A. Fröhlich in Plauen.
Am 15.2.1920 war Begrüßung.\\
7. März: Kränzel im Wettinschlösschen.\\
18. März: Tetangsomg im Immungshaus.\\
21. März: Kränzel im Hanschwitz.


\newpage
%%% Seite 3 %%%


\end{document}
